% Last Modified 12/22/2013

\documentclass[11pt,oneside]{article}

\usepackage{geometry}
\usepackage{xspace}
\usepackage{url}
\usepackage{comment}

\newcommand{\buflo} {BuFLO\xspace}
\newcommand{\csbuflo} {Congestion-Sensitive BuFLO\xspace}
\newcommand{\csb} {CS-BuFLO\xspace}

\geometry{letterpaper,tmargin=1in,bmargin=1in,lmargin=1in,rmargin=1in,headheight=0in,headsep=0in,footskip=.3in}

\begin{document}

\title{\textbf{Research Statement}}
\author{\textbf{Xiang Cai}}
\date{}

\maketitle

\section{Introduction}

The Internet has become an important part in many people's everyday life, and
it has also changed the way people used to live. We fetch news by browsing
websites rather than reading the newspaper, chat with friends on social
networks, and manage bank accounts online, etc. We embrace the great
conveniences these services bring us, but at the same time, we have to face the
inevitable problems that come with them -- network security and user privacy.
There are many cyber attacks that can jeopardize online security and privacy.
For example, phishing websites can trick a user to reveal his passwords, social
engineering attacks may link a user's online accounts to his real life identity,
			etc.

One such attack is called a website fingerprinting attack \cite{hintz-pets02},
	which enables an adversary to infer which website a victim is visiting,
	even if the victim uses an encrypting proxy, such as Tor
	\cite{tor-website}. Encrypting transferred data can hide the data itself,
	but still reveals other information, e.g. packet sizes, timing, and
	directions of packets, etc. In a website fingerprinting attack, an
	adversary analyzes these features, and attempts to infer the web page being
	visited by a victim. This attack scenario only requires an adversary to be
	able to eavesdrop, which is hard to be detected by a victim, yet can cause
	serious privacy issues. It has been shown that web page fingerprinting
	attacks are possible against many privacy services, including IPSec
	tunnels, SSH tunnels, and Tor. If mounted by internet censors, the attacks
	clearly can affect millions of users. Many researchers have stepped into
	the battlefield of developing website fingerprinting attacks and defenses.
	It is interesting to know that who will eventually win this battle, and what are the
	prices the winner has to pay. E.g. whether a defender has to suffer a huge
	bandwidth cost to prevent information leakages, or an attacker needs more
	computing power than feasible to get what he wants. My research tries to
	answer these questions by thoroughly analyzing this type of attack, and
	building real attack and defense systems to evaluate their impacts in
	reality.

\section{Previous and Ongoing Work}

Several researchers have developed web page fingerprinting attacks on encrypted
web traffic, as occurs when the victim uses HTTPS, link-level encryption, such
as WPA, or an encrypting tunnel such as SSH, a VPN, or IPSec. Herrmann, et al.,
   used a Multinomial Naive Bayes classifier on features that captured only
   packet sizes, and achieved over 94\% success in recognizing packet traces
   from a set of 775 possible web pages visited using SSH
   \cite{herrmann-ccsw09}. Panchenko, et al., used ad hoc, HTTP-specific
   features with support vector machines to achieve a 54.61\% success rate on
   the same data set \cite{panchenko-wpes11}. On the other hand, Dyer et al.
   performed a thorough survey of past attacks and past network-level defenses
   and found that no network-level defense was secure \cite{dyer-snp12}. 

In order to build a secure and efficient defense system, we need to understand
why existing attacks can achieve high success rates and where existing defenses
fall short. After investigating several existing attacks, we found that most attacks
focus on packet sizes, and many throw away all information about packet
ordering. Packet sizes do carry a lot of information in these scenarios, where
data packets are simply padded to a multiple of the block size (typically 16
		bytes), but Tor pads all data packets to a multiple of 512 bytes,
	 providing much less information. However, the attacker can observe more
	 things from the traffic other than just packet sizes, e.g. the order of
	 packets exchanged between the user and the proxy. We believe a better
	 attacker exists if he utilizes all what he observed.

% dlsvm
Web pages can consist of multiple objects, such as HTML files, images, and
flash objects, and browsers send separate requests for each object. There is
some inherent stability in the ordering of requests: browsers cannot request an
object until they have received the portion of a page that references it. These
facts suggest a simple representation for the attacker's traffic observations,
	  and a similarity metric the attacker can use to compare traces.  We
	  developed a web site fingerprinting attack DLSVM \cite{cai-ccs12}, which
	  represents a trace of $\ell$ packets as a vector $t=(d_1, \ldots,
			  d_\ell)$, where $d_i=\pm s_i$, where $s_i$ is the size of the
	  $i$th packet and the sign indicates the direction of the packet.  Our
	  attack compares traces $t$ and $t'$ using the Damerau-Levenshtein edit
	  distance~\cite{navarro-acmcs01}, which is the length of the shortest
	  sequence of character insertions, deletions, substitutions, and
	  transpositions required to transform $t$ into $t'$. In the context of our
	  packet traces, these edits correspond to packet and request re-ordering,
	  request omissions (e.g. due to caching), and slight variations in the
	  sizes of requests and responses. Thus, this model and distance metric are
	  a good match for real network and HTTP-level behavior. To build a
	  classifier for recognizing encrypted, anonymized page loads of 1 of $n$
	  web pages, an attacker collects $k$ traces of each page, using the same
	  privacy system, e.g. Tor or an SSH proxy, in use by the victim.  He then
	  trains a support vector machine~\cite{vapnik-svm95} using a kernel based
	  on edit distance. Evaluation shows that our attack significantly
	  outperforms other proposed attacks on these and other defenses. Our
	  attack can determine which web page, out of 100 possibilities, a victim
	  is visiting with over 80\% success rate.

% csbuflo
The success of DLSVM implies that an effective defense should hide all possible
side channel information leakages an attacker can possibly observe. Dyer, et
al., described \buflo, a hypothetical defense scheme that hides all information
about a website, except possibly its size, and performed a simulation-based
evaluation that found that, although \buflo is able to offer good security, it
incurs a high cost to do so. We built \csbuflo (\csb), an extension to \buflo
that includes numerous security and efficiency improvements.  \csb represents a
new approach to the design of fingerprinting defenses.  Most
previously-proposed defenses were designed in response to known attacks, and
therefore took a black-listing approach to information leaks, i.e. they tried
to hide specific features, such as packet sizes.  In designing \csb, we take a
white-listing approach -- we start with a design that hides all traffic
features, and iteratively refine the design to reveal certain traffic features
that enable us to achieve significant performance improvements without
significantly harming security. 

\csb remedies several drawbacks of \buflo.
\begin{itemize}

\item
\buflo does not adapt when the user is visiting fast or slow websites. It
wastes bandwidth when loading slow sites, and causes large latency when loading
fast websites.

-- We developed a rate adaptation algorithm which adapts the
transmission rate to math the rate of the sender.

\item
\buflo is not TCP-friendly.

-- \csb uses TCP to be congestion friendly, and
uses feedback from the TCP stack in order to reduce the amount of junk data it
needs to send.

\item
\buflo either completely hides the size of the website or completely reveals it
($\pm d$ bytes).  Thus it does not provide the same level of security to all
websites. It also has large overheads for small websites.  Thus its overhead is
also unevenly distributed.

-- \csb uses a scale-independent padding scheme and
monitors the state of the page loading process to avoid some unnecessary
overheads.

\end{itemize}

We also developed bounds on the trade-off between security and bandwidth overhead that
any fingerprinting defense scheme can achieve. This enables us to compare
schemes with different security/overhead trade-offs by comparing how close they
are to optimal. Our experiments find that \csb is closer to the optimal
bandwidth/security trade-off curve than Tor or plain SSH.

\section{Future Research}

Our theoretical analysis of lower bounds suggests that the reason website
fingerprinting defenses are expensive is not because websites are so different.
It is because the defense lacks the knowledge of where to put the cover traffic
correctly, thus it has to put it everywhere. An interesting direction for
future research is to build a defense that can remember information about
websites it has seen before, thus it has sufficient knowledge to make a wise
decision about where to put the cover traffic.

Suppose websites $w_1,\dots,w_n$ are all static and constructed such that
loading each site requires performing a fixed, serialized sequence of requests
and responses, e.g. each web page contains a javascript program that loads
objects one at a time in a fixed order.  Let $s_i[j]=1$ iff the $j$th byte that
must be transmitted to load page $w_i$ is a transmission in the upstream
direction.  

Loading website $w_i$ via a deterministic defense mechanism produces a fixed
trace $t_i$.  Consider a DLSVM attacker that converts $t_i$ into a binary
string $z_i$ where $z_i[j]=1$ iff the $j$th byte of $t_i$ is an upstream byte.
Since, for these websites, the defense mechanisms cannot delete or re-order
bytes, we must have that $s_i$ is a sub-sequence of $z_i$.

A defense can achieve perfect security only if $z_1=\dots=z_n$.  Thus the
defense system must compute some string, $z$, that is simultaneously a
super-sequence of $s_1,\dots,s_n$.  Minimizing the cost of such a defense is
thus equivalent to finding the shortest common super-sequence (SCS) of $s_1,
	 \dots, s_n$.  This problem is NP-hard\cite{SCS_SR8}.

However, there is a simple 2-approximation for the binary SCS problem.  Let
$\ell$ be the length of the longest string $s_1,\dots,s_n$.  Their SCS must be
at least $\ell$ long, but is at most $2\ell$ long, since every binary string of
length at most $\ell$ is a sub-sequence of $(01)^\ell$.  Thus for any set of
static websites $w_1,\dots,w_n$, there exists a deterministic offline defense
that achieves $\epsilon$-security against DLSVM-style attackers and incurs
bandwidth cost that is at most twice the bandwidth of the largest website under
consideration. The drawback of the simple algorithm is that it does not hide
inter-packet intervals. We need to develop a new algorithm for computing SCS
that hides both sizes and time intervals, and to evaluate the overhead it
brings.

There are some other interesting topics related to website fingerprinting
defenses that worth pursuing. To name a few:

\begin{itemize}

\item
Even if we have a perfect defense scheme that hides web site contents, it is
still possible that the defense system can be easily recognized, i.e. a signature
of the defense system itself is easily obtained. Hence, a censor can choose to
block any traffic pattern matching that signature. Hiding the defense system
itself can be a challenging task.

\item
The threat model of website fingerprinting attacks involves an attacker
monitoring the first hop traffic of the victim. This requirement can be
achieved relatively easily in reality, compared to other models that require an
active attacker or involve monitoring multiple-hop traffic. However, in a
different network model, e.g. Content-Centric Networking
\cite{jacobson-CoNEXT09}, the traffic observed on the first hop may not reflect
the real inter-packet intervals due to the caching scheme used in CCN. It is
not clear whether the new network architecture is resilient to website
fingerprinting attacks by nature.

\end{itemize} 

\bibliographystyle{plain}
\bibliography{xiang}

\end{document}


