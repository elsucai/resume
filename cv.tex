% Xiang Cai's Resume
% Created: Oct. 22 2010
% Last Modified: Dec. 7 2013

\documentclass[11pt,oneside]{article}
\usepackage{geometry}
\usepackage{verbatim}
\usepackage{xspace}
\usepackage[T1]{fontenc}
\usepackage[colorlinks=false,linkcolor=black]{hyperref}
\def\pdfBorderAttrs{/Border [0 0 0] } 
\pagestyle{empty}
\geometry{letterpaper,tmargin=1in,bmargin=1in,lmargin=1in,rmargin=1in,headheight=0in,headsep=0in,footskip=.3in}

\setlength{\parindent}{0in}
\setlength{\parskip}{0in}
\setlength{\itemsep}{0in}
\setlength{\topsep}{0in}
\setlength{\tabcolsep}{0in}

% Name and contact information
\newcommand{\name}{Xiang Cai\xspace}
\newcommand{\addrlinea}{29 Millbrook Drive\xspace}
\newcommand{\addrlineb}{Stony Brook, NY 11790, USA\xspace}
\newcommand{\phone}{\hfill \textit{Mobile: }(631)680-8544\xspace}
\newcommand{\email}{\hfill \textit{Email: }xcai@cs.stonybrook.edu\xspace}
\newcommand{\cvphone}{\textit{(631)680-8544}\xspace}
\newcommand{\cvemail}{\textit{xcai@cs.stonybrook.edu}\xspace}


%%%%%%%%%%%%%%%%%%%%%%%%%%%%%%%%%%%%%%%%%%%%%%%%%%%%%%%%%
% New commands and environments

% This defines how the name looks
\newcommand{\bigname}[1]{
	\begin{center}\fontfamily{phv}\selectfont\Huge\scshape#1\end{center}
}

% This defines contact info style
\newcommand{\contact}[1]{
	\begin{flushleft}\fontfamily{phv}\selectfont\Huge\Large#1\end{flushleft}
}


% A ressection is a main section (<H1>Section</H1>)
\newenvironment{ressection}[1]{
	\vspace{4pt}
	{\fontfamily{phv}\selectfont\Large#1}
	\begin{itemize}
	\vspace{3pt}
}{
	\end{itemize}
}

% A resitem is a simple list element in a ressection (first level)
\newcommand{\resitem}[1]{
	\vspace{-4pt}
	\item \begin{flushleft} #1 \end{flushleft}
}

% A ressubitem is a simple list element in anything but a ressection (second level)
\newcommand{\ressubitem}[1]{
	\vspace{-1pt}
	\item \begin{flushleft} #1 \end{flushleft}
}

% A resbigitem is a complex list element for stuff like jobs and education:
%  Arg 1: Name of company or university
%  Arg 2: Location
%  Arg 3: Title and/or date range
\newcommand{\resbigitem}[3]{
	\vspace{-5pt}
	\item
	\textbf{#1} --- #2 \\
	\textit{#3}
}

% This is a list that comes with a resbigitem
\newenvironment{ressubsec}[3]{
	\resbigitem{#1}{#2}{#3}
	\vspace{-2pt}
	\begin{itemize}
}{
	\end{itemize}
}

% Same as ressubsec, but no item is required
\newenvironment{ressubsec_empty}[3]{
	\resbigitem{#1}{#2}{#3}
	\vspace{-2pt}
}


% This is a simple sublist
\newenvironment{reslist}[1]{
	\resitem{\textbf{#1}}
	\vspace{-5pt}
	\begin{itemize}
}{
	\end{itemize}
}


%%%%%%%%%%%%%%%%%%%%%%%%%%%%%%%%%%%%%%%%%%%%%%%%%%%%%%%%%
% Now for the actual document:

\begin{document}

\fontfamily{ppl} \selectfont

\begin{comment}
% begin coverletter
\bigname{\name}

\vspace{-8pt} \rule{\textwidth}{1pt}

\vspace{8 pt}

\begin{center}
\addrlinea \\
\addrlineb \\
\cvphone \\
\end{center}

\vspace{30pt}

December 12, 2013

\vspace{8pt}
Recruitment Team

\textit{Company, Inc.}

\textit{address}

\vspace{30pt}
Dear Hiring Executive,

\vspace{8pt}
I am writing in response to your recent advertisement for a Research Scientist
position (job id \#12345) on your website. Based on my background and experience, I believe I am
a strong candidate for the position.

\vspace{8pt}

\begin{table}[h]
\centering
\begin{tabular}{p{10cm}p{1cm}p{10cm}}
\textbf{YOUR NEEDS}	&	&	\textbf{MY QUALIFICATION}	\\
a	&	&	aa	\\
b	&	&	bb	\\

\end{tabular}
\end{table}

\vspace{8pt}
Thank you for your considerations. If you would like to discuss further about
my qualifications, please do not hesitate to contact me via email at 
\cvemail  or via telephone at \cvphone. I look forward to
talking with you!

\vspace{8pt}
Sincerely yours,

\vspace{30pt}
Xiang Cai

\vspace{30pt}
\textit{Attached: curriculum vitae}

\newpage

% end of coverletter
\end{comment}

% Name with horizontal rule
\bigname{\name}

\vspace{-8pt} \rule{\textwidth}{1pt}

%\vspace{-1pt} {\small\itshape \addr \hfill \phone; \email}

\vspace{8 pt}

\contact{Contact Information}

\begin{table}[h]
\begin{tabular}[h]{p{0.5\textwidth}p{0.5\textwidth}}
\addrlinea & \phone \\
\addrlineb & \email \\
\end{tabular}
\end{table}

%\begin{ressection}{Objective}
%
%	\resitem{Obtain a full time job in the area of Software Engineering}
%
%\end{ressection}
%%%%%%%%%%%%%%%%%%%%%%%%
\begin{ressection}{Education}

	\begin{ressubsec_empty}{Stony Brook University}{Stony Brook, NY}{\textbf{Ph.D.} candidate
		in Computer Science(GPA:3.93/4) \hfill \textbf{Expected May 2014}}	
	
		Advisor: Prof. Rob Johnson
		
%		\ressubitem{Core Courses:
%			\begin{small}
%				\textit{Operating Systems; Analysis of Algorithms; Theory of
%					Computation; Theory of Database Systems; Fundamental of
%						Computer Networks} 
%			\end{small}
%		}
%		\ressubitem{Candidate Exam Report:
	%		\href{http://docs.google.com/fileview?id=0Byne4VTOYpXPMGFjMzM5NWQtZWNiMC00M2Y2LThiNjktOTdmNjk4ODY2N2Rm&hl=en}
	%		{
%			\begin{small}
%				\textit{Type based Security Verification and Program Transformation}
%			\end{small}
%			}
%		}                                                                     
	\end{ressubsec_empty}

	\vspace{2pt}

	\begin{ressubsec_empty}{University of Sciences \& Technology of
		China(USTC)}{Hefei, Anhui, China}{\textbf{B.E.} in Computer
			Science(GPA:3.83/4.3) \hfill \textbf{July 2007}}
	\end{ressubsec_empty}


\end{ressection}

\begin{ressection}{Research Interests}
	\resitem{
		System Security, Network Security and Privacy.
	}
\end{ressection}


%\begin{ressection}{Technical Skills}
%	\resitem{
%		Programming Languages: \textbf{C/C++} (proficient, 3+ years of experience), \textbf{Java} (intermediate), \textbf{python, ruby, PHP} (beginner)
%	}
%	\resitem{
%		Development Tools: \textbf{gdb}, \textbf{git}, \textbf{Netbeans} 
%	}
%	\resitem{
%		Operating Systems: \textbf{Linux (Ubuntu)} (4+ years of experience)
%	}
%
%	\resitem{Broad and diversified background in \textbf{System Programming},
%		\textbf{Algorithms}, \textbf{Software Development} and
%			\textbf{Computer Security}. Specialization in \textbf{System
%				Security} and \textbf{Network Security}. Excellent troubleshooting/problem solving skills.
%	}

%\end{ressection}


\begin{ressection}{Working Experience}
	\begin{ressubsec}{Hewlett-Packard Labs}{Princeton, NJ}{Research Associate Internship \hfill \textbf{June 2012 -- August 2012}}
		\begin{small}
			\resitem{Conducted large scale data analysis on passive DNS traffic to find malicious command-and-control servers.}
			\resitem{Implemented a prototype system for detecting malicious URLs using Java.}
		\end{small}
	\end{ressubsec}
\end{ressection}


\begin{ressection}{Research Projects}

	\resitem{\textbf{Website fingerprinting defenses. Aug.
		2012.} 
		\begin{small}
		
		Developed bounds on the trade-off between security and
		bandwidth overhead that any fingerprinting defense scheme can
		achieve, which enables us to compare schemes with different
		security/overhead trade-offs by comparing how close they are
		to the lower bound. 
		
		Refine, implement, and evaluate
		the Congestion-Sensitive BuFLO scheme, which attracted the attention of the
		Tor developers. Experiments find that Congestion-
		Sensitive BuFLO has high overhead (around 2.3-2.8x) but can
		get 6x closer to the bandwidth/security trade-off lower bound
		than Tor or plain SSH.
		\end{small}				
	}

	\resitem{\textbf{Website fingerprinting attacks on Tor. May.
		2011.} 
		\begin{small}
		
		Proposed and implemented two new website fingerprinting
			attacks on the Tor anonymity system. Our web page
			classifier was novel in that, unlike all previous
			classifiers, it completely ignored packet sizes. Despite
			its simplicity, our attack outperformed 
			recently-published fingerprinting attacks on Tor. Source code: \textbf{git@github.com:xiang-cai/fingerprinting\_attack\_code.git}
		\end{small}				
	}

	\resitem{\textbf{A novel machine registration procedure. Dec.
		2010.}
		\begin{small}
		
		Designed a novel online machine
			registration system. Users can register their computers to
			certain service providers (e.g. Banks, SNS websites) by
			visiting the registration website and then following specific
			instructions. This new process is as simple as most existing
			machine registration procedures yet more secure. Phishing and MITM attacks against the
			system are almost impossible.
		\end{small}	
	}
			
	\resitem{\textbf{A C library for serializable file-system
				accessing. Feb. 2010.} 
		\begin{small}
		
		Proposed and implemented an
					easy-to-use, portable, provably-secure system for
					accessing UNIX file-systems without race conditions
					and that supported arbitrary sequences of operations
					within each pseudo-transaction and which had
					negligible overhead on a mail-server
					benchmark. Source code: \textbf{git@github.com:xiang-cai/trace.git}
		\end{small}
	}
	
	\resitem{\textbf{Race attack on Unix File-Systems. Oct. 2008.}
		\begin{small}
		
		Defeated two proposed Unix file-system race
			condition defense mechanisms. We argued that all
			kernel-based dynamic race detectors must have a model of
			the programs they protect or provide imperfect
			protection. Source code: \textbf{git@github.com:xiang-cai/race\_attack.git}
		\end{small}
	} 


%	\begin{ressubsec}{Stony Brook University, Computer Science
%		Department}{Stony Brook, NY}{Research Assistant under \textbf{Prof. Rob
%			Johnson}: Summer 2008--present}
%		\begin{small}	
%
%			\resitem{\textbf{Websites fingerprinting attacks on Tor. May.
%				2011}. 
%				\begin{small}
%				Proposed and implemented two new website fingerprinting
%					attacks on the Tor anonymity system. Our web page
%					classifier was novel in that, unlike all previous
%					classifiers, it completely ignored packet sizes. Despite
%					its simplicity, our attack outperformed 
%					recently-published fingerprinting attacks on Tor.
%					(C++/Java/ruby/MYSQL)
%				\end{small}				
%			}
%
%			\resitem{\textbf{A novel machine registration procedure. Dec.
%				2010}. 
%				\begin{small}
%				Designed a novel online machine
%					registration system. Users can register their computers to
%					certain service providers (e.g. Banks, SNS websites) by
%					visiting the registration website and then following specific
%					instructions. This new process is as simple as most existing
%					machine registration procedures yet more secure. Phishing and MITM attacks against the
%					system are almost impossible. (PHP/MYSQL)
%				\end{small}	
%			}
%					
%			\resitem{\textbf{A C library for serializable file-system
%						accessing. Feb. 2010}. 
%				\begin{small}
%				Proposed and implemented an
%							easy-to-use, portable, provably-secure system for
%							accessing UNIX file-systems without race conditions
%							and that supported arbitrary sequences of operations
%							within each pseudo-transaction and which had
%							negligible overhead on a mail-server
%							benchmark.(C/C++/Shell)
%				\end{small}
%			}
%			
%			\resitem{\textbf{Race attack on Unix File-Systems. Oct. 2008}.
%				\begin{small}
%				Defeated two proposed Unix file-system race
%					condition defense mechanisms. We argued that all
%					kernel-based dynamic race detectors must have a model of
%					the programs they protect or provide imperfect
%					protection.(C/Shell)
%				\end{small}
%			} 
%		\end{small}
%	\end{ressubsec}


%	\begin{ressubsec}{University of Sciences \& Technology of China}{National
%		High Performance Computing Center , Hefei, Anhui, China}{Undergraduate
%			Students Research Program, under \textbf{Prof. Liusheng Huang}:
%				July. 2006--Oct. 2006}
%		\begin{small}
%			\resitem{\textbf{Explored existing MAC protocols for Wireless Sensor Networks.}}
%	    	\resitem{\textbf{Optimized the performance of existing protocols under certain conditions.}}
%		\end{small}
%	\end{ressubsec}

\end{ressection}


%%%%%%%%%%%%%%%%%%%%%%%%
\begin{comment}
\begin{ressection}{Selected Course Projects}

	\resitem{\textbf{Very Simple Real File-System, Dec. 2008. [C/Kernel Programming]} 
		\begin{small}
		Implemented a very simple file-system with a group of three people, which supported regular file operations such as create, open, close, lookup, link, delete, etc.
		\end{small}
	}

	\resitem{\textbf{Linux Stackable File-system, Nov. 2008. [C/Kernel Programming]} 
		\begin{small}
		Implemented a stackable file system in Linux with a group of three people, which supported transparent
			file integrity checking.
		\end{small}
	}

	\resitem{\textbf{Adding A System Call To The Linux Kernel, Oct. 2008. [C/Kernel Programming]}
		\begin{small}
		Implemented a system call by adding a module to the Linux kernel (2.6.3.26), which can insert and delete data at user
			specified positions within a regular file.
		\end{small}		
	}

	\resitem{\textbf{WEP Cracking Project, Dec. 2007. [Java/C]} 
		\begin{small}
		Hacked and broke into a
		128 bit WEP protected wireless network by sniffing the headers of the packets being transmitted.
		\end{small}
	}
\end{ressection}
\end{comment}

\begin{ressection}{Talks}
	\resitem{\textbf{Touching From a Distance: Website Fingerprinting
		Attacks and Defenses.} \begin{small}
		
		\textit{Invited talk to Symantec Research Labs. September 25th, 2012}. \textit{Conference presentation at ACM CCS, Raleigh, NC, October 2012}.
		\end{small}}

	\resitem{\textbf{Exploiting Unix File-System Races via Algorithmic
		Complexity Attacks.} \begin{small}
		
		\textit{Conference presentation at IEEE Symposium on Security and Privacy, Oakland, CA, May 2009}.
		\end{small}}
\end{ressection}

\begin{ressection}{Academic Activities}
	\resitem{\textbf{Security and Communication Networks 2013, } \begin{small}Invited journal reviewer.\end{small}
	}
\end{ressection}

\begin{ressection}{Publications}
	\resitem{New Approaches to Website Fingerprinting Defenses. \begin{small} \textbf{Xiang Cai}, Rishab Nithyanand
			and Rob Johnson. \textit{In submission}.
	\end{small}}

	\resitem{Touching From a Distance: Website Fingerprinting
		Attacks and Defenses. \begin{small} \textbf{Xiang Cai}, Xincheng Zhang
			and Rob Johnson. \textit{ACM Conference on Computer and Communications Security, Raleigh, NC, October 2012}. (Acceptance rate: 19\%, 81/426)\end{small}}

	\resitem{Fixing Races For Good: Serializable File-System
		Access for UNIX. \begin{small} \textbf{Xiang Cai}, Rucha Lale,
		Xincheng Zhang and Rob Johnson. Recommended for publication. \textit{ACM Transactions on Storage, 2012}.\end{small}}

	\resitem{Exploiting Unix File-System Races via Algorithmic
		Complexity Attacks. \begin{small} \textbf{Xiang Cai}, Yuwei Gui, and
			Rob Johnson. \textit{IEEE Symposium on Security and Privacy,
				Oakland, CA, May 2009}. (Acceptance rate: 10.2\%, 26/254)\end{small}}

\end{ressection}

\begin{ressection}{References}
	\resitem{Available upon request.
	}
\end{ressection}

%%%%%%%%%%%%%%%%%%%%%%%%


\begin{comment}
%%%%%%%%%%%%%%%%%%%%%%%%
\begin{ressection}{Awards}
\resitem{\begin{small}Sep. 2007 New Graduate students Fellowship, Stony Brook University.\end{small}}
\resitem{\begin{small}Oct. 2006, Citigroup-CSTS Excellent Scholarship for undergraduate students.\end{small}}
\resitem{\begin{small}Oct. 2005, San Pu scholarship for undergraduate students.\end{small}}
\resitem{\begin{small}October 2004, Outstanding Student Scholarship (Grade 2), USTC.\end{small}}
\resitem{\begin{small}Nov. 2003, Second place in the English Debate Competition of USTC.\end{small}}
\end{ressection}
\end{comment}

\end{document}
